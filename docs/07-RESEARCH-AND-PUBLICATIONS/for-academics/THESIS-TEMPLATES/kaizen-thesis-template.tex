% Kaizen Thesis Template for Mobius Systems Research
% Version: 1.0.0
% Cycle: C-151
% Description: LaTeX template for academic papers on Mobius Systems

\documentclass[12pt,a4paper]{article}

% ============================================
% PACKAGES
% ============================================
\usepackage[utf8]{inputenc}
\usepackage[T1]{fontenc}
\usepackage{amsmath,amssymb,amsthm}
\usepackage{graphicx}
\usepackage{hyperref}
\usepackage{booktabs}
\usepackage{algorithm}
\usepackage{algorithmic}
\usepackage{listings}
\usepackage{xcolor}
\usepackage{natbib}
\usepackage{geometry}
\usepackage{fancyhdr}
\usepackage{lipsum}

% ============================================
% DOCUMENT SETTINGS
% ============================================
\geometry{margin=1in}
\setlength{\parindent}{0pt}
\setlength{\parskip}{1em}

% ============================================
% CUSTOM COLORS
% ============================================
\definecolor{mobiusviolet}{RGB}{139, 92, 246}
\definecolor{mobiusindigo}{RGB}{99, 102, 241}
\definecolor{codeblue}{RGB}{30, 64, 175}
\definecolor{codegray}{RGB}{107, 114, 128}

% ============================================
% CODE LISTING STYLE
% ============================================
\lstset{
    basicstyle=\ttfamily\small,
    keywordstyle=\color{codeblue}\bfseries,
    commentstyle=\color{codegray}\itshape,
    stringstyle=\color{mobiusviolet},
    numbers=left,
    numberstyle=\tiny\color{codegray},
    breaklines=true,
    frame=single,
    captionpos=b
}

% ============================================
% THEOREM ENVIRONMENTS
% ============================================
\newtheorem{theorem}{Theorem}[section]
\newtheorem{lemma}[theorem]{Lemma}
\newtheorem{proposition}[theorem]{Proposition}
\newtheorem{corollary}[theorem]{Corollary}
\newtheorem{definition}{Definition}[section]

% ============================================
% CUSTOM COMMANDS
% ============================================
\newcommand{\MIC}{\textsc{MIC}}
\newcommand{\MII}{\textsc{MII}}
\newcommand{\KS}{\textsc{KS}}
\newcommand{\Mobius}{M\"{o}bius}

% ============================================
% HEADER/FOOTER
% ============================================
\pagestyle{fancy}
\fancyhf{}
\rhead{Mobius Systems Research}
\lhead{\leftmark}
\rfoot{Page \thepage}
\lfoot{Cycle C-151}

% ============================================
% TITLE INFORMATION
% ============================================
\title{
    \vspace{-1cm}
    \textcolor{mobiusviolet}{\rule{\linewidth}{0.5mm}} \\[0.4cm]
    \Huge \textbf{[Your Thesis Title Here]} \\[0.2cm]
    \large Investigating [Topic] in Mobius Systems \\[0.3cm]
    \textcolor{mobiusviolet}{\rule{\linewidth}{0.5mm}}
}

\author{
    [Author Name] \\
    \textit{[University/Institution]} \\
    \texttt{[email@institution.edu]}
}

\date{
    [Month Year] \\
    \vspace{0.3cm}
    \small Submitted in partial fulfillment of the requirements for \\
    [Degree Type] in [Field]
}

% ============================================
% DOCUMENT BEGIN
% ============================================
\begin{document}

\maketitle
\thispagestyle{empty}
\newpage

% ============================================
% ABSTRACT
% ============================================
\begin{abstract}
\noindent
[Write your abstract here. The abstract should be 150-300 words and include:
(1) The problem or question addressed,
(2) The methods used,
(3) Key findings, and
(4) Significance of the work.
Mention Mobius Integrity Index (MII), Mobius Integrity Credits (MIC), or other relevant concepts.]

\vspace{0.5cm}
\noindent\textbf{Keywords:} Mobius Systems, Integrity Index, MIC Currency, AI Safety, [Additional Keywords]
\end{abstract}
\newpage

% ============================================
% TABLE OF CONTENTS
% ============================================
\tableofcontents
\newpage

% ============================================
% INTRODUCTION
% ============================================
\section{Introduction}

\subsection{Background}
The development of advanced artificial intelligence systems presents unprecedented challenges for governance, economics, and social coordination. Traditional approaches to these challenges—regulation, market mechanisms, and institutional design—may be insufficient for the speed and scale of AI development.

Mobius Systems represents a novel approach to these challenges through its Continuous Integrity Architecture. At the core of this system is the \textbf{Mobius Integrity Index (MII)}, a composite measure of system health:

\begin{equation}
    \MII = 0.25 \cdot M + 0.20 \cdot H + 0.30 \cdot I + 0.25 \cdot E
\end{equation}

where:
\begin{itemize}
    \item $M$ = Memory (documentation, test coverage)
    \item $H$ = Human (oversight, review indicators)
    \item $I$ = Integrity (security, no violations)
    \item $E$ = Ethics (charter compliance, virtue tags)
\end{itemize}

\subsection{Research Questions}
This thesis addresses the following research questions:
\begin{enumerate}
    \item [Research Question 1]
    \item [Research Question 2]
    \item [Research Question 3]
\end{enumerate}

\subsection{Contributions}
The main contributions of this work are:
\begin{enumerate}
    \item [Contribution 1]
    \item [Contribution 2]
    \item [Contribution 3]
\end{enumerate}

% ============================================
% LITERATURE REVIEW
% ============================================
\section{Literature Review}

\subsection{AI Safety and Alignment}
[Review relevant literature on AI safety, alignment, and governance...]

\subsection{Cryptoeconomics and Incentive Design}
[Review literature on token economics, mechanism design, and incentive structures...]

\subsection{Integrity Measurement}
[Review approaches to measuring trust, integrity, and system health...]

% ============================================
% METHODOLOGY
% ============================================
\section{Methodology}

\subsection{Formal Framework}

\begin{definition}[Mobius Integrity Credit]
A \textbf{Mobius Integrity Credit (MIC)} is a unit of account minted when an integrity action $a$ increases the system MII above threshold $\tau$:
\begin{equation}
    \MIC_{\text{minted}} = \alpha \cdot \max(0, \MII - \tau) \cdot v(a)
\end{equation}
where $\alpha = 1.0$ is the base coefficient and $v(a)$ is the value of action $a$.
\end{definition}

\begin{definition}[Kaizen Shard]
A \textbf{Kaizen Shard (KS)} is the smallest subdivision of MIC:
\begin{equation}
    1 \text{ MIC} = 1,000,000 \text{ KS}
\end{equation}
\end{definition}

\subsection{Data Collection}
[Describe your data collection methodology...]

\subsection{Analysis Approach}
[Describe your analytical approach...]

% ============================================
% RESULTS
% ============================================
\section{Results}

\subsection{Empirical Findings}
[Present your empirical findings...]

\begin{table}[h]
\centering
\caption{Example Results Table}
\begin{tabular}{@{}lccc@{}}
\toprule
Metric & Baseline & Mobius & Improvement \\
\midrule
Integrity Score & 0.72 & 0.96 & +33\% \\
Participation Rate & 45\% & 78\% & +73\% \\
Cost Efficiency & 1.0 & 0.65 & -35\% \\
\bottomrule
\end{tabular}
\end{table}

\subsection{Theoretical Results}

\begin{theorem}[Integrity Convergence]
Under mild assumptions, the MII converges to a stable equilibrium above threshold $\tau$ as the number of participants $n \to \infty$.
\end{theorem}

\begin{proof}
[Proof here...]
\end{proof}

% ============================================
% DISCUSSION
% ============================================
\section{Discussion}

\subsection{Interpretation of Results}
[Interpret your findings in the context of the research questions...]

\subsection{Implications}
[Discuss theoretical and practical implications...]

\subsection{Limitations}
[Acknowledge limitations of your study...]

% ============================================
% CONCLUSION
% ============================================
\section{Conclusion}

\subsection{Summary}
[Summarize key findings...]

\subsection{Future Work}
[Suggest directions for future research...]

\subsection{Closing Remarks}
As the Mobius principle states: \textit{"We heal as we walk."} This research contributes to the ongoing effort to build systems that enable human and artificial intelligence to coordinate safely and beneficially.

% ============================================
% ACKNOWLEDGMENTS
% ============================================
\section*{Acknowledgments}
[Thank advisors, collaborators, funding sources, and Mobius Systems community...]

% ============================================
% REFERENCES
% ============================================
\bibliographystyle{plainnat}
\bibliography{references}

% If no .bib file, use manual entries:
\begin{thebibliography}{99}

\bibitem{mobius2025}
Kaizen Cycle. (2025). \textit{Mobius Systems: Continuous Integrity Architecture}. 
Technical Documentation, Version C-151.

\bibitem{ktt2025}
[Author]. (2025). The Kaizen Turing Test: A Framework for Evaluating AI Integrity.
\textit{Journal of AI Safety Research}.

\bibitem{mic2025}
[Author]. (2025). MIC Tokenomics: Integrity-Backed Currency for Civic Coordination.
\textit{Proceedings of the Workshop on Cryptoeconomics}.

\end{thebibliography}

% ============================================
% APPENDICES
% ============================================
\appendix
\section{Technical Specifications}

\subsection{MII Calculation Algorithm}
\begin{lstlisting}[language=Python, caption={MII Calculation}]
def calculate_mii(components):
    """
    Calculate Mobius Integrity Index
    
    Args:
        components: dict with keys M, H, I, E (each 0-1)
    
    Returns:
        float: MII score (0-1)
    """
    weights = {'M': 0.25, 'H': 0.20, 'I': 0.30, 'E': 0.25}
    
    mii = sum(
        weights[k] * components[k] 
        for k in weights
    )
    
    return max(0.0, min(1.0, mii))
\end{lstlisting}

\subsection{MIC Minting Formula}
\begin{lstlisting}[language=Python, caption={MIC Minting}]
KS_PER_MIC = 1_000_000
THRESHOLD = 0.95
ALPHA = 1.0

def mint_mic(mii, shard_value=1.0):
    """
    Mint MIC based on MII score
    
    Returns:
        tuple: (mic, ks)
    """
    if mii <= THRESHOLD:
        return (0, 0)
    
    delta = mii - THRESHOLD
    mic = ALPHA * delta * shard_value
    ks = round(mic * KS_PER_MIC)
    
    return (mic, ks)
\end{lstlisting}

\section{Additional Data}
[Include supplementary data, tables, figures...]

\end{document}
