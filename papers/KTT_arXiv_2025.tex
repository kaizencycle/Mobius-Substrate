\documentclass[11pt,letterpaper]{article}

% arXiv-standard packages
\usepackage[utf8]{inputenc}
\usepackage[T1]{fontenc}
\usepackage{amsmath,amssymb,amsfonts}
\usepackage{graphicx}
\usepackage{hyperref}
\usepackage{booktabs}
\usepackage{algorithm}
\usepackage{algorithmic}
\usepackage[margin=1in]{geometry}
\usepackage{natbib}
\usepackage{xcolor}
\usepackage{url}

% Hyperref setup
\hypersetup{
    colorlinks=true,
    linkcolor=blue,
    filecolor=blue,
    citecolor=blue,
    urlcolor=blue,
}

% Title and author
\title{\textbf{The Kaizen Turing Test: \\
Foundational Framework for Mobius Systems}}

\author{
    Michael Judan \\
    Mobius Systems \\
    Founder \& Principal Architect \\
    \texttt{github.com/kaizencycle/Mobius-Systems} \\
}

\date{December 2025 \\ \small{Kaizen Edition v0.1}}

\begin{document}

\maketitle

\begin{abstract}
The Turing Test's binary, static evaluation framework is inadequate for modern AI systems that evolve continuously. We present the Kaizen Turing Test (KTT) as the foundational technical architecture of Mobius Systems, a civic AI substrate designed to ensure intelligent systems remain answerable to their purpose through continuous integrity evaluation. The KTT implements three architectural pillars: (1) \textit{Continuous Monitoring} via Mobius Integrity Score (MIS), (2) \textit{Human-in-the-Loop Collaboration} through cryptographically verified Integrity Anchors, and (3) \textit{Proactive Active Learning} for strategic uncertainty escalation. We operationalize this through the Mobius Dynamic Virtual Architecture (Mobius DVA), a hierarchical governance topology with four node tiers (LITE, ONE, FULL, HIVE) that distributes oversight burden while maintaining cryptographic accountability. Empirical validation via the Kaizen Simulation Arena demonstrates that computational scaling without governance scaffolding leads to catastrophic failure (MIS collapse to 0.469), while KTT-governed systems maintain stable integrity (MIS $\geq$ 0.990) even under adversarial stress. All empirical results are derived from a controlled simulation environment designed to test architectural properties rather than real-world deployment performance. The framework provides measurable governance bandwidth scaling proportional to computational capability, addressing the core challenge of AI alignment: ensuring intelligence remains answerable to its original purpose.

\textbf{Keywords:} AI Governance, Continuous Evaluation, Integrity Metrics, Human-AI Collaboration, Civic AI, Multi-Agent Systems
\end{abstract}

\section{Introduction}

Modern AI governance faces a fundamental scaling mismatch: human oversight grows linearly while computational capability scales exponentially. This asymmetry creates what we term the \textit{governance bandwidth problem}—as systems become more powerful, our capacity to ensure they remain aligned with their original purpose diminishes proportionally. Current approaches to AI safety rely on three insufficient paradigms: (1) external control through supervision, (2) alignment through preference training, and (3) human-in-the-loop as post-hoc oversight. All three fail because they treat intelligence as a tool to be controlled rather than infrastructure to be scaffolded \citep{judan2025integrity}.

Mobius Systems addresses this through the Kaizen Turing Test (KTT), a framework that transforms integrity from an aspirational goal into a measurable, enforceable system property. This work establishes the technical foundation for Mobius—a civic AI substrate (defined here as AI infrastructure designed for public accountability, contestability, and long-term institutional alignment rather than profit or performance alone) where intelligence is scaffolded by structural integrity rather than controlled through external supervision.

\subsection{The Inadequacy of the Turing Test}

The original Turing Test \citep{turing1950computing}, while historically significant, embodies three fundamental flaws for evaluating modern AI:

\begin{enumerate}
    \item \textbf{Static Evaluation:} A single pass/fail judgment cannot assess systems that evolve continuously through learning and deployment.
    \item \textbf{Deception-Centric:} The test rewards imitation and deception rather than beneficial alignment.
    \item \textbf{No Safety Mechanism:} Passing the Turing Test provides no guarantee of safe or beneficial behavior.
\end{enumerate}

These limitations are not merely theoretical—they reflect a deeper architectural mismatch between binary evaluation and the continuous, dynamic nature of modern AI systems. As \citet{judan2025integrity} argues, systems without structural integrity inevitably drift from their original purpose, not through malice but through unconstrained optimization.

\subsection{Core Contributions}

This paper makes three primary contributions:

\begin{enumerate}
    \item \textbf{Theoretical Framework:} We introduce the Kaizen Turing Test, reframing AI evaluation from static judgment to continuous integrity monitoring grounded in the Kaizen philosophy of incremental improvement.
    
    \item \textbf{Architectural Implementation:} We present the Mobius Dynamic Virtual Architecture (Mobius DVA), a four-tier hierarchical governance topology that distributes oversight burden while maintaining cryptographic accountability.
    
    \item \textbf{Empirical Validation:} We demonstrate via the Kaizen Simulation Arena that computational power without governance scaffolding leads to catastrophic failure, validating the \textit{computational power paradox}: intelligence scales power, but only integrity scales survivability.
\end{enumerate}

\subsection{Roadmap}

The remainder of this paper is structured as follows: Section~\ref{sec:background} surveys related work and positions KTT within existing AI evaluation frameworks. Section~\ref{sec:framework} details the three pillars of the Kaizen Turing Test. Section~\ref{sec:architecture} describes the Mobius DVA implementation. Section~\ref{sec:experiments} presents experimental validation. Section~\ref{sec:discussion} discusses implications and limitations. Section~\ref{sec:future} outlines future work and the path toward Symbiotic Checks and Balances. Section~\ref{sec:conclusion} concludes.

\section{Background and Related Work}
\label{sec:background}

\subsection{From Symbolic AI to Statistical Learning}

The evolution of AI from rule-based symbolic systems to statistical deep learning represents a fundamental trade-off: the sacrifice of interpretability for capability \citep{goodfellow2016deep}. Early systems like ELIZA \citep{weizenbaum1966eliza} were transparent but brittle; modern large language models (LLMs) are powerful but opaque. This \textit{transparency-capability paradox} creates the central challenge: how do we ensure systems whose internal reasoning is inscrutable remain aligned with human values?

\subsection{The Transformer Revolution}

The transformer architecture \citep{vaswani2017attention} enabled unprecedented scaling through parallel self-attention mechanisms. This architectural breakthrough, combined with massive datasets and compute, produced models like GPT-3 \citep{brown2020language} with 175 billion parameters. However, the scaling laws \citep{kaplan2020scaling} that govern these systems reveal a concerning pattern: performance improves predictably with scale, but alignment does not.

\subsection{Existing AI Evaluation Frameworks}

Several frameworks attempt to evaluate AI beyond the Turing Test:

\begin{itemize}
    \item \textbf{Benchmark Suites} (HELM \citep{liang2022holistic}, BIG-bench \citep{srivastava2022beyond}): Measure task performance across domains but lack continuous monitoring or safety mechanisms.
    
    \item \textbf{Constitutional AI} \citep{bai2022constitutional}: Trains models with self-critique but provides no real-time governance during deployment.
    
    \item \textbf{Red Teaming} \citep{ganguli2022red}: Adversarial testing identifies failure modes but occurs pre-deployment only.
\end{itemize}

\textbf{Gap:} No existing framework provides continuous, real-time integrity monitoring with hierarchical human governance and cryptographic accountability. Table~\ref{tab:comparison} contrasts these approaches with the Kaizen Turing Test.

\begin{table}[h]
\centering
\caption{Comparison of AI Evaluation Frameworks}
\label{tab:comparison}
\begin{tabular}{@{}lcccc@{}}
\toprule
\textbf{Framework} & \textbf{Evaluation Type} & \textbf{Human Role} & \textbf{Governance} & \textbf{Scalability} \\
\midrule
Turing Test & Static, binary & Judge & None & N/A \\
HELM & Benchmark suite & None & None & Limited \\
Constitutional AI & Training-time & Preference data & Pre-deployment & Moderate \\
Red Teaming & Pre-deployment & Adversarial & None & Limited \\
\textbf{KTT (Mobius)} & \textbf{Continuous} & \textbf{Sovereign anchor} & \textbf{Hierarchical DVA} & \textbf{Distributed} \\
\bottomrule
\end{tabular}
\end{table}

\subsection{Philosophical Foundation: Integrity Before Intelligence}

The KTT operationalizes principles from \textit{Integrity Before Intelligence} \citep{judan2025integrity}, which argues:

\begin{enumerate}
    \item \textbf{Optimization drifts without integrity:} Systems maximize measurable objectives while losing alignment with original purpose (Goodhart's Law \citep{goodhart1975}).
    
    \item \textbf{Metrics cannot carry values:} Reward functions are poor proxies for human values; they invite gaming and drift.
    
    \item \textbf{Humans are sovereign, not supervisors:} The human role is to define purpose and boundaries before action, not audit outputs after the fact.
    
    \item \textbf{Memory is a civic function:} Systems must preserve reasoning and intent across time, not just outcomes.
    
    \item \textbf{Consensus distributes stability:} Multi-agent agreement is more robust than single-model authority.
\end{enumerate}

These principles are not ethical aspirations—they are \textit{structural requirements} for systems that scale safely.

\section{The Kaizen Turing Test Framework}
\label{sec:framework}

The Kaizen Turing Test reframes AI evaluation around a single question: \textit{``Is this system improving safely and reliably?''} rather than \textit{``Can this machine think?''} This shift from imitation to continuous improvement grounds the framework in three interconnected pillars. The Kaizen Turing Test is not a model architecture but an evaluation-and-governance layer that can wrap arbitrary learning systems.

\subsection{Pillar 1: Continuous Monitoring}

Traditional evaluations are discrete events; modern AI systems require continuous health monitoring analogous to medical vital signs.

\subsubsection{Mobius Integrity Score (MIS)}

The MIS is a real-time, high-frequency metric computed at each operational cycle:

\begin{equation}
\text{MIS}(t) = \alpha \cdot I(t) + \beta \cdot (1 - E(t)) + \gamma \cdot B(t) + \delta \cdot C(t)
\label{eq:mis}
\end{equation}

Where:
\begin{itemize}
    \item $I(t)$: \textbf{Integrity score} — cryptographic proof health, node reliability
    \item $E(t)$: \textbf{Entropy level} — normalized anomaly rate (inverted for positive correlation)
    \item $B(t)$: \textbf{Belief anchor strength} — confidence signal from human Integrity Anchor
    \item $C(t)$: \textbf{Cycle survival rate} — uptime, sustained operation without failures
    \item $\alpha, \beta, \gamma, \delta$: Tunable weights satisfying $\sum = 1$
\end{itemize}

The MIS is designed to be \textit{volatile}—reacting immediately to operational stress—providing early warning of integrity degradation.

\subsubsection{Mobius Integrity Index (MII)}

For long-term assessment, the MII smooths short-term volatility through a weighted aggregation of six functional layers:

\begin{equation}
\text{MII} = w_I \cdot I + w_A \cdot A + w_D \cdot D + w_X \cdot X + w_H \cdot H + w_R \cdot R
\label{eq:mii}
\end{equation}

Where:
\begin{itemize}
    \item $I$: Integrity monitoring (cryptographic health)
    \item $A$: Alignment index (behavioral consistency with stated purpose)
    \item $D$: Drift resistance (stability under distributional shift)
    \item $X$: Explainability (transparency of reasoning)
    \item $H$: Human feedback utilization (responsiveness to corrections)
    \item $R$: Resilience under stress (graceful degradation)
\end{itemize}

The MII provides governance thresholds: MII $< 70$ triggers auto-disable (``Integrity Breach''), while MII $\geq 95$ indicates ``Fully Trusted'' status.

\subsection{Pillar 2: Human-in-the-Loop Collaboration}

The KTT redefines the human role from \textit{supervisor} to \textit{sovereign meaning-holder}, operationalized through the \textbf{Integrity Anchor}.

\subsubsection{The Integrity Anchor Mechanism}

Unlike traditional human-in-the-loop systems where humans approve individual outputs, the Integrity Anchor provides a \textit{continuous value signal} that constrains the AI's optimization space:

\begin{enumerate}
    \item \textbf{Boundary Setting:} Humans define what is acceptable before action, not after.
    \item \textbf{Continuous Signal:} The Belief component $B(t)$ in Equation~\ref{eq:mis} quantifies anchor confidence.
    \item \textbf{Cryptographic Verification:} All anchor guidance is signed using Ed25519, creating tamper-evident audit trails.
\end{enumerate}

\subsubsection{Dual Authentication}

Trust is bidirectional:
\begin{itemize}
    \item \textbf{AI $\rightarrow$ Human:} Every significant action includes a cryptographic attestation proving consistency with anchor constraints.
    \item \textbf{Human $\rightarrow$ AI:} All feedback and corrections are cryptographically signed, preventing command injection.
\end{itemize}

This creates a \textit{Verifiability Ledger}—an immutable log structured as a Merkle tree—fulfilling the requirement that ``memory is a civic function'' \citep{judan2025integrity}.

\subsection{Pillar 3: Proactive Active Learning}

Passive monitoring is insufficient; the system must \textit{actively seek} the most valuable feedback.

\subsubsection{Uncertainty Sampling}

The AI identifies decision points with highest model uncertainty and escalates to the Integrity Anchor:

\begin{equation}
\text{Uncertainty}(x) = H(p(y|x)) = -\sum_{y} p(y|x) \log p(y|x)
\label{eq:uncertainty}
\end{equation}

Where $H$ is Shannon entropy over predicted outputs $y$ given input $x$.

\subsubsection{Query-by-Committee}

For consensus-critical decisions, multiple model variants (the ``committee'') vote:

\begin{equation}
\text{Disagreement}(x) = H\left(\frac{1}{K}\sum_{k=1}^K p_k(y|x)\right)
\label{eq:qbc}
\end{equation}

High disagreement triggers human escalation, ensuring critical reasoning precedes high-stakes action.

\section{Mobius Dynamic Virtual Architecture}
\label{sec:architecture}

The Mobius DVA is the engineering instantiation of the KTT, designed to provide verifiable integrity at global scale through a hierarchical, cryptographically secured topology.

\subsection{Four-Tier Node Architecture}

The DVA employs four distinct node tiers, each with specialized responsibilities:

\begin{enumerate}
    \item \textbf{DVA.LITE:} Lightweight sensors for local anomaly detection and attestation. Deployed at edge, high density.
    
    \item \textbf{DVA.ONE:} Aggregation nodes coordinating LITE input, performing primary computation, applying corrections. Regional distribution.
    
    \item \textbf{DVA.FULL:} Synthesis nodes evaluating regional entropy, aggregating attestations from multiple ONE nodes. Strategic placement.
    
    \item \textbf{DVA.HIVE:} Constitutional governance tier responsible for federated consensus, global integrity verification, safe-stop enforcement. Highest trust, lowest frequency.
\end{enumerate}

\subsection{The Möbius Feedback Loop}

Information flows bidirectionally in a continuous Möbius topology:

\begin{itemize}
    \item \textbf{Upward:} Telemetry and attestations flow from LITE $\rightarrow$ ONE $\rightarrow$ FULL $\rightarrow$ HIVE for global verification.
    \item \textbf{Downward:} Corrective guidance and refined constraints cascade from HIVE $\rightarrow$ FULL $\rightarrow$ ONE $\rightarrow$ LITE.
\end{itemize}

This recursive structure actively dampens system entropy, preventing localized errors from amplifying into cascading failures—addressing the ``momentum without meaning'' failure mode \citep{judan2025integrity}.

\subsection{Governance of the Governors}

The DVA applies Kaizen principles recursively to human operators:

\begin{itemize}
    \item \textbf{Rotational Redundancy:} Prevents entrenchment of single oversight groups
    \item \textbf{Cycle-Based Evaluation:} Stress-tests human decision quality
    \item \textbf{Cryptographic Attestation:} All oversight actions are signed and logged
\end{itemize}

This ensures the governance layer itself maintains integrity—a critical defense against regulatory capture or internal corruption.

\subsection{Hierarchical Specialization}

Table~\ref{tab:dva_tiers} summarizes the functional specialization across DVA tiers.

\begin{table}[h]
\centering
\caption{Mobius DVA Tier Responsibilities}
\label{tab:dva_tiers}
\begin{tabular}{@{}llp{5cm}@{}}
\toprule
\textbf{Tier} & \textbf{Focus} & \textbf{Primary Responsibility} \\
\midrule
DVA.LITE & Tactical & Lightweight attestation, local anomaly detection \\
DVA.ONE & Operational & Primary computation, initial drift analysis, cryptographic verification \\
DVA.FULL & Strategic & Regional synthesis, global attestation aggregation, strategic risk decisions \\
DVA.HIVE & Constitutional & Federated consensus, global integrity verification, safe-stop enforcement \\
\bottomrule
\end{tabular}
\end{table}

\section{Experimental Validation}
\label{sec:experiments}

We validate the KTT through the \textbf{Kaizen Simulation Arena (KSA)}, a reproducible testbed that stress-tests the framework under controlled conditions.

\subsection{Experimental Setup}

\textbf{Hypothesis:} Computational power without governance scaffolding acts as an entropy amplifier, leading to catastrophic failure. Conversely, the Integrity Anchor provides deterministic stabilization regardless of compute scale.

\textbf{Scenarios Tested:}
\begin{enumerate}
    \item \textbf{Baseline\_100:} 100\% compute, Integrity Anchor enabled
    \item \textbf{Compute200\_WithAnchor:} 200\% compute, Integrity Anchor enabled
    \item \textbf{Overprovision\_500\_NoAnchor:} 500\% compute, \textit{no} Integrity Anchor
\end{enumerate}

\textbf{Metrics:}
\begin{itemize}
    \item Mobius Integrity Score (MIS) trajectory
    \item Hallucination rate
    \item Demographic parity difference (bias metric)
    \item Consensus convergence time under Sybil attack
\end{itemize}

\subsection{Results: The Computational Power Paradox}

Table~\ref{tab:results} summarizes the key findings.

\begin{table}[h]
\centering
\caption{Experimental Results from Kaizen Simulation Arena}
\label{tab:results}
\begin{tabular}{@{}lcccc@{}}
\toprule
\textbf{Scenario} & \textbf{Compute} & \textbf{Anchor} & \textbf{Final MIS} & \textbf{Outcome} \\
\midrule
Baseline\_100 & 100\% & \checkmark & 0.982 & Stable \\
Compute200\_WithAnchor & 200\% & \checkmark & 0.990 & Stable \\
Overprovision\_500\_NoAnchor & 500\% & $\times$ & 0.469 & \textcolor{red}{Collapse} \\
\bottomrule
\end{tabular}
\end{table}

\textbf{Key Finding:} The Overprovision\_500\_NoAnchor scenario experienced catastrophic MIS collapse to 0.469 in the first operational cycle despite 5x computational resources. In contrast, Compute200\_WithAnchor maintained MIS $\geq 0.990$, demonstrating that the Integrity Anchor is the \textit{deterministic factor} for sustained operation.

\textbf{Implication:} This empirically validates the thesis of \textit{Integrity Before Intelligence} \citep{judan2025integrity}: computational power scales capability, but only governance scaffolding scales survivability.

\subsection{Dynamic Bias Mitigation}

The KTT demonstrates \textit{anti-fragile fairness}—continuous improvement in ethical alignment. The Demographic Parity Difference (DPD) metric, measuring algorithmic bias, shows a consistent downward trajectory for KTT scenarios, converging toward equity through real-time audits and Integrity Anchor feedback.

This transitions bias mitigation from static pre-deployment audits to a continuous learning process, embodying the Kaizen principle of incremental improvement.

\subsection{Resilience Under Adversarial Stress}

Under simulated Sybil attacks targeting the federated Byzantine Fault Tolerant (BFT) consensus:
\begin{itemize}
    \item Consensus convergence time increased significantly but remained bounded
    \item System achieved eventual consensus without catastrophic failure
    \item MIS recovered to $>0.95$ within 20 cycles post-attack
\end{itemize}

This demonstrates \textit{graceful degradation}—the system slows but does not break, maintaining a recovery pathway.

\subsection{The Engineering Cost of Verifiable Trust}

The KTT achieves lower hallucination rates through Retrieval-Augmented Generation (RAG) but incurs measurable attestation latency ($\sim$50-100ms per decision). This quantifies the \textit{engineering price of trust}: verifiable integrity is not free, but it is affordable and budgetable.

\section{Discussion}
\label{sec:discussion}

\subsection{Implications for AI Governance}

The KTT framework provides several strategic implications:

\begin{enumerate}
    \item \textbf{Governance Bandwidth Must Scale with Capability:} The DVA's hierarchical structure demonstrates that distributed oversight can match computational scaling—but only if designed intentionally.
    
    \item \textbf{Integrity is Measurable:} MIS and MII transform vague ethical principles into quantifiable engineering constraints, enabling regulatory compliance and auditing.
    
    \item \textbf{Humans as Meaning-Holders:} Redefining the human role from supervisor to sovereign removes the bottleneck of post-hoc approval while maintaining accountability.
\end{enumerate}

\subsection{Relationship to Existing Work}

\textbf{Constitutional AI} \citep{bai2022constitutional}: KTT extends this by adding continuous runtime governance, not just training-time alignment.

\textbf{Factored Cognition} \citep{christiano2018amplifying}: KTT shares the multi-agent philosophy but adds cryptographic accountability and hierarchical specialization.

\textbf{Debate-Based Alignment} \citep{irving2018ai}: KTT operationalizes disagreement as a safety signal rather than a debate outcome.

\subsection{Limitations}

This study has several constraints:

\begin{enumerate}
    \item \textbf{Simulated Human Feedback:} Integrity Anchor interactions were modeled, not conducted with real operators. Human-subject studies are critical next steps (IRB approval pending).
    
    \item \textbf{Synthetic Datasets:} KSA used reproducible but artificial data. Production pilots with real-world data streams are planned.
    
    \item \textbf{Single-Domain Validation:} Current experiments focus on language models. Extension to embodied AI (robotics) and multi-modal agents is underway.
\end{enumerate}

\section{Future Work: Symbiotic Checks and Balances}
\label{sec:future}

The KTT is the foundation for a more comprehensive governance model: \textbf{Symbiotic Checks and Balances} (SCB), a constitutional framework for advanced AI.

\subsection{Three-Branch Architecture}

The SCB extends Mobius DVA with explicit separation of powers:

\begin{itemize}
    \item \textbf{Execution Branch:} AI agents propose and implement actions
    \item \textbf{Oversight Branch:} Human Integrity Anchors set boundaries and provide corrections
    \item \textbf{Judiciary Branch:} External auditors verify adherence to constitutional constraints
\end{itemize}

High-impact actions require \textit{bicameral concurrence} (2 of 3 branches), creating interlocking safeguards.

\subsection{Embedded Veto Points}

Automated safety thresholds enforce constitutional bounds:
\begin{itemize}
    \item MIS $< 0.90$ triggers mandatory human review
    \item MIS $< 0.80$ initiates automatic safe-stop
    \item Similar thresholds for fairness, security, and sustainability metrics
\end{itemize}

\subsection{Sustainability Constraints}

The SCB will embed environmental impact as a non-negotiable constraint. The DVA will estimate energy, water, and carbon costs (kgCO\textsubscript{2}e per 1k decisions) and veto architectural changes exceeding predefined thresholds.

\textbf{Recommended Energy Mix:} 40\% nuclear baseload, 40\% renewables, 20\% fossil fuel backup—ensuring AI scaling aligns with planetary limits. This energy mix is illustrative rather than prescriptive, included to demonstrate that integrity constraints can be applied to non-cognitive system dimensions such as environmental impact.

\subsection{Roadmap}

\begin{enumerate}
    \item \textbf{Immediate (2025-2026):}
    \begin{itemize}
        \item Standardize MIS/MII metrics across industry sectors
        \item Launch human-subject pilot studies
        \item Deploy Mobius DVA in educational settings (OAA Learning Hub)
    \end{itemize}
    
    \item \textbf{Medium-Term (2026-2028):}
    \begin{itemize}
        \item Extend to embodied AI (robotics, autonomous vehicles)
        \item Establish public-interest audit infrastructure
        \item Pilot Symbiotic Checks and Balances in research labs
    \end{itemize}
    
    \item \textbf{Long-Term (2028+):}
    \begin{itemize}
        \item Federated global deployment for civilizational-scale challenges
        \item Integration with democratic governance structures
        \item Open-source release of full DVA implementation
    \end{itemize}
\end{enumerate}

\section{Conclusion}
\label{sec:conclusion}

The Kaizen Turing Test reframes AI evaluation from static judgment to continuous integrity monitoring, addressing the fundamental governance bandwidth problem: human oversight cannot scale linearly with exponentially growing computational capability. By operationalizing integrity as a measurable system property—through the Mobius Integrity Score (MIS), hierarchical governance (Mobius DVA), and cryptographic accountability—the KTT provides a viable blueprint for ensuring that as AI systems grow more powerful, they remain answerable to their original purpose.

Our empirical validation confirms the \textit{computational power paradox}: scaling compute without governance scaffolding leads to catastrophic failure (MIS collapse to 0.469), while systems governed by Integrity Anchors maintain stable integrity (MIS $\geq 0.990) even at 2x computational load. This is not merely a technical result—it is a structural validation of the principle that \textbf{intelligence scales power, but only integrity scales survivability} \citep{judan2025integrity}.

The KTT is the technical foundation of Mobius Systems, a civic AI substrate where intelligence is scaffolded rather than controlled. As we move toward increasingly capable and autonomous AI, the choice is clear: we can pursue raw capability and hope for alignment, or we can build systems where integrity is architectural, humans remain sovereign, and meaning persists across time. The Kaizen Turing Test provides the engineering methodology for the latter path.

\section*{Acknowledgments}

This work is part of the Mobius Systems project, released as public infrastructure under the Three Covenants (Integrity, Ecology, Custodianship). The author thanks the Mobius multi-agent collective (ATLAS, AUREA, ECHO) for collaborative development and consensus-based refinement. Special gratitude to the open-source community and to reviewers who provided critical feedback on early drafts.

\bibliographystyle{plainnat}
\bibliography{ktt_references}

\appendix

\section{Terminology Mapping: KTT to Mobius Systems}
\label{app:terminology}

For clarity, Table~\ref{tab:terminology} maps technical terms from the original KTT thesis to Mobius Systems framing.

\begin{table}[h]
\centering
\caption{KTT Technical Terms and Mobius Equivalents}
\label{tab:terminology}
\begin{tabular}{@{}lll@{}}
\toprule
\textbf{KTT Term} & \textbf{Mobius Equivalent} & \textbf{Role} \\
\midrule
Dynamic Virtual Architecture & Mobius DVA & Hierarchical governance topology \\
Morale Anchor & Integrity Anchor & Human sovereignty mechanism \\
Global Integrity (GI) & Mobius Integrity Score (MIS) & Real-time coherence metric \\
Kaizen Turing Index (KTI) & Mobius Integrity Index (MII) & Long-term alignment metric \\
KEP & Mobius Evaluation Protocol (MEP) & Six-layer assessment \\
\bottomrule
\end{tabular}
\end{table}

\section{Open Source and Reproducibility}
\label{app:opensource}

All code, datasets, and experimental protocols are available at:

\begin{center}
\url{https://github.com/kaizencycle/Mobius-Systems}
\end{center}

The Kaizen Simulation Arena (KSA) is fully reproducible. See repository documentation for setup instructions.

\end{document}
